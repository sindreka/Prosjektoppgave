%%%%%%%%%%%%%%%%%%%%%%%%%%%%%%%%%%%%%%%%%%%%%%%%%%%%%%%%%%%%%%%%%%%%%%%%%%%%%%%%%%%%%%%%%%%%%%%%%%%%%%%%%%%%%%%%%%%%%%
\chapter{Discussion and conclusion}%%%%%%%%%%%%%%%%%%%%%%%%%%%%%%%%%%%%%%%%%%%%%%%%%%%%%%%%%%%%%%%%%%%%%%%%%%%%%%%%%%
%%%%%%%%%%%%%%%%%%%%%%%%%%%%%%%%%%%%%%%%%%%%%%%%%%%%%%%%%%%%%%%%%%%%%%%%%%%%%%%%%%%%%%%%%%%%%%%%%%%%%%%%%%%%%%%%%%%%%%
!!!!!!!!!!!!!!!!!!TING SOM MÅ MED i rekkefølge!!!!!!!!!!!!!!!!!!!!!!!!!!!\\
\begin{itemize}
\item teoretisk kompleksitet
\item og minnebruk
\item convergens
\item restart variabel
\item og $\gamma$.
\item $\delta$, $\epsilon$, $\gamma$
\item speedup
\item Sammenligningen mellom de to metoder
\end{itemize}

Theoretically all methods perform about the same when $p$ is separable. If $p$ is not separable DM has a clear advantage. If we assume $\gamma \propto m^2/n^2$, which we concluded in section \ref{sec:rrest}.


%Regarding convergence all methods can perform equally well, the drawback is the need to choose an appropriate $\delta$ for KPM and KPM$(n)$. With a larger $\delta$ KPM and KPM$(n)$ is less accurate, but with smaller $\delta$ we might restart too many times, making the methods inefficient. 


%The theoretical results shows that the only real candidate to outperform DM is KPM$(n)$, KPM is only near when $p$ is separable. This also holds for memory requirements. In practice it is much the same. KPM did perform overall worst, while KPM$(n)$ and DM was close to each other. KPM$(n)$ is asymptotically better if $\rho$ is large, $n$ is chosen appropriate and $p$ is separable. If $p$ was not separable DM was better.

%It seams that the optimal restart variable is when $n = \rho$. If $\gamma \propto m^2/n^2$ then KPM$(\rho)$ and DM has the same asymptotic complexity, but KPM$(\rho)$ uses less memory. Solving KPM$(n)$ with separable $p$ is faster than solving DM, so with several processing units KPM$(\rho)$ will be better because KPM$(n)$ works very well with many processing units. 

The reason for the high parallel performance is the natural independence in the method. The only communication needed between processors is when adding results, this can be done in $\log_2(\texttt{nP})$ additions. Note that a good restart variable is also a good value to use with parallel computations. \\

It is worth noting that DM did not work when $\rho>300$ due to memory shortage, while KPM$(n)$ had no problem with $\rho = 1000$ and $n = 40$. \\




%%%%%%%%%%%%%%%%%%%%%%%%%%%%%%%%%%%%%%%%%%%%%%%%%%%%%%%%%%%%%%%%%%%%%%%%%%%%%%%%%%%%%%%%%%%%%%%%%%%%%%%%%%%%%%%%%%%%%%
\chapter*{Further work}%%%%%%%%%%%%%%%%%%%%%%%%%%%%%%%%%%%%%%%%%%%%%%%%%%%%%%%%%%%%%%%%%%%%%%%%%%%%%%%%%%%%%%%%%%%%%%
%%%%%%%%%%%%%%%%%%%%%%%%%%%%%%%%%%%%%%%%%%%%%%%%%%%%%%%%%%%%%%%%%%%%%%%%%%%%%%%%%%%%%%%%%%%%%%%%%%%%%%%%%%%%%%%%%%%%%%
To obtain better results KPM should be implemented in a more parallel friendly language, as for example \texttt{C}. It would also be a benefit to use a large computer to get data with larger $\rho$.
%KPM should be implement in a more parallel friendly language, like \texttt{C}, with a larger computer to get better results regarding parallel computations and larger systems. 
KPM should also be implemented for other function than the heat equation. 
%There should also be a test where $\delta$ and $ n $ increases with $\rho$ or $k$ to see how this impacts the results.
%%%%%%%%%%%%%%%%%%%%%%%%%%%%%%%%%%%%%%%%%%%%%%%%%%%%%%%%%%%%%%%%%%%%%%%%%%%%%%%%%%%%%%%%%%%%%%%%%%%%%%%%%%%%%%%%%%%%%%
\chapter*{My code}%%%%%%%%%%%%%%%%%%%%%%%%%%%%%%%%%%%%%%%%%%%%%%%%%%%%%%%%%%%%%%%%%%%%%%%%%%%%%%%%%%%%%%%%%%%%%%%%%%%
%%%%%%%%%%%%%%%%%%%%%%%%%%%%%%%%%%%%%%%%%%%%%%%%%%%%%%%%%%%%%%%%%%%%%%%%%%%%%%%%%%%%%%%%%%%%%%%%%%%%%%%%%%%%%%%%%%%%%%

If you are interested in any of the code used here you can find it at: \\
\emph{https://github.com/sindreka/Prosjektoppgave}




%\vspace{5cm}
\newpage
\begin{thebibliography}{9}

    \bibitem{elena}
    
    E. Celledoni, I. Moret
    \emph{A Krylov projection method for system of ODEs}
    \emph{Applied Numerical Mathematics 23 (1997) 365-378}
    \emph{http://www.sciencedirect.com/science/article/pii/S0168927497000330}
    \bibitem{trap}
    
    \emph{https://en.wikipedia.org/wiki/Trapezoidal\_rule}

    \bibitem{arnold}
    Yousef Saad,
    \emph{Iterative methods for sparse linear systems, second edition},
    Page 154,
    Algorithm 6.1,
    2003
    \bibitem{krylovprop}
    Yousef Saad,
    \emph{Iterative methods for sparse linear systems, second edition},
    Page 154,
    Proposition 6.5,
    2003
        \bibitem{numop}
    Yousef Saad,
    \emph{Iterative methods for sparse linear systems, second edition},
    Page 160,
    2003
        \bibitem{kryconv}
    Yousef Saad,
    \emph{Iterative methods for sparse linear systems, second edition},
    Page 171, 
    Proposition 6.10,
    2003
        \bibitem{parpool}
    
    \emph{http://se.mathworks.com/help/distcomp/parpool.html}
    \bibitem{parfor}
    
    \emph{http://se.mathworks.com/help/matlab/ref/parfor.html}
    \bibitem{complex}
    
    \emph{http://en.wikipedia.org/wiki/Computational\_complexity\_of\_mathematical\_operations}
    
    
\end{thebibliography}
\end{document}
