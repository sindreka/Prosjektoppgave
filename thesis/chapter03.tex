%%%%%%%%%%%%%%%%%%%%%%%%%%%%%%%%%%%%%%%%%%%%%%%%%%%%%%%%%%%%%%%%%%%%%%%%%%%%%%%%%%%%%%%%%%%%%%%%%%%%%%%%%%%%%%%%%%%%%%
\chapter{Implementation}%%%%%%%%%%%%%%%%%%%%%%%%%%%%%%%%%%%%%%%%%%%%%%%%%%%%%%%%%%%%%%%%%%%%%%%%%%%%%%%%%%%%%%%%%%%%%%
%%%%%%%%%%%%%%%%%%%%%%%%%%%%%%%%%%%%%%%%%%%%%%%%%%%%%%%%%%%%%%%%%%%%%%%%%%%%%%%%%%%%%%%%%%%%%%%%%%%%%%%%%%%%%%%%%%%%%%
\label{sec:imp}
In this chapter the implementation of the methods will be explained. Discretizing in space and time will be done in section \ref{sec:space} and \ref{sec:time} respectively.
In section \ref{sec:not} we present what to measure, and how, together with some information about computers and programs used to generate data.
In section \ref{sec:test} some test problems are stated.

%%%%%%%%%%%%%%%%%%%%%%%%%%%%%%%%%%%%%%%%%%%%%%%%%%%%%%%%%%%%%%%%%%%%%%%%%%%%%%%%%%%%%%%%%%%%%%%%%%%%%%%%%%%%%%%%%%%%%%
\section{Discretization in space} \label{sec:space}
%%%%%%%%%%%%%%%%%%%%%%%%%%%%%%%%%%%%%%%%%%%%%%%%%%%%%%%%%%%%%%%%%%%%%%%%%%%%%%%%%%%%%%%%%%%%%%%%%%%%%%%%%%%%%%%%%%%%%%
Consider the spacial square $[0,1] \times [0,1]$ and divide each direction into $\rho+2$ pieces, with each piece having the length $h_s = 1/(\rho+1)$. Since the boundary conditions are known, there are only $\rho$ unknown numbers in each direction. Let $x_i = h_s \cdot i$ and $y_j = h_s \cdot j$, with $i,j =1,2,\cdots, \rho + 1 $. 
%We need to find $v$ and $P(t)$ in an ordered fashion. 
Let  $v_{i+\rho j} = g(x_i, y_j)$, and $P(t)_{i+\rho j} = p(t,x_i, y_j)$, where $i,j = 1,2,\cdots, \rho+1$.
The Laplacian will be approximated by the five point formula given as the matrix $A$, 
\begin{equation} \label{eqn:fivepoint} 
\begin{aligned} 
A = \frac{1}{h_s^2} 
\begin{bmatrix}
T & I & & &\\
I& T & I & &\\
& \ddots & \ddots & \ddots & \\
& & I& T & I\\
& & & I & T
\end{bmatrix}
, T  &= 
\begin{bmatrix}
-4 & 1 & & &\\
1 & -4 & 1 & &  \\
& \ddots & \ddots & \ddots & \\
&  & 1 & -4 & 1 \\
 & & & 1 & -4
\end{bmatrix},
I &= 
\begin{bmatrix}
1 & &\\
& \ddots & \\
& & 1 \\
\end{bmatrix}
\end{aligned}\cite{fivepoint}.
\end{equation} 
This is a second order approximation.
Notice that $m = \rho ^2$.

%%%%%%%%%%%%%%%%%%%%%%%%%%%%%%%%%%%%%%%%%%%%%%%%%%%%%%%%%%%%%%%%%%%%%%%%%%%%%%%%%%%%%%%%%%%%%%%%%%%%%%%%%%%%%%%%%%%%%%
\section{Discretisation in time} \label{sec:time}
%%%%%%%%%%%%%%%%%%%%%%%%%%%%%%%%%%%%%%%%%%%%%%%%%%%%%%%%%%%%%%%%%%%%%%%%%%%%%%%%%%%%%%%%%%%%%%%%%%%%%%%%%%%%%%%%%%%%%%
Consider the time domain $t \in [0,1] $, and divide it in $k$ pieces, giving each piece the length $h_t = 1/(k-1)$. Let $t_l = h_t\cdot l$, with $l = 0,1,\cdots,k-1 $.\\

Algorithm \ref{alg:fullkry}, \ref{alg:restkry} and equation \ref{eqn:DI} all contain a differential equation in time.
The trapezoidal rule, 
\begin{equation} \label{eqn:trap}
\int \limits_a^b f(t) dt \approx \frac{h}{2} \sum \limits_{l = 1}^N(f(t_{l+1})+f(t_l)) \cite{trap}
\end{equation}
will be used to solve these equations. 
The iteration scheme will only be derived for equation \eqref{eqn:numheat}, but 
this it is easily generalizable to the other differential equations discussed.
To obtain the iteration scheme, write $q$ instead of $f$, use equation \eqref{eqn:heat}, and insert the numerical simplifications from earlier in this section and from section \ref{sec:space}, that is
\begin{equation}
q(t_{l+1}) - q(t_l) = \int \limits_{t_l}^{t_{l+1}} \frac{d q}{d t} dt \approx \frac{h}{2}(A q(t_{l+1})+f(t_{l+1})v +A q(t_l)+f(t_l) v) .
\end{equation}
Solving for $q(t_{l+1})$ gives the Crank-Nicholson scheme for the heat equation:
\begin{equation} \label{eqn:trapscheme}
\begin{aligned}
q(t_{l+1}) \approx (I-h_t/2 A)^{-1}(q(t_l) + \frac{h_t}{2}( A q(t_{l}) + f(t_l)v+f(t_{l+1})v))\\
%U^{l+1} = (I-h_t/2 A)^{-1}(U^l + \frac{h_t}{2}( A U^l + G^l+G^{l+1}))
\end{aligned}
\end{equation} 
This is a second order method.
Because of this discretizing, $P(t)$ and $q(t)$ will be saved as $m \times k $ matrices, and $f(t)$ as a $k$ vector, in the points $t_l$ with $l = 0,1,\cdots,k-1 $ . An implementation of equation \eqref{eqn:trapscheme} is given in algorithm \ref{alg:integrate}.\\
\begin{algorithm} 
\begin{algorithmic} \caption{Integration} \label{alg:integrate}  
\STATE Start with $A$, $f(t)$, $v$, $h_t$ and $k$
\STATE $invmat = (I-h_t/(2A))^{-1}$
\STATE $q(t_1) = invmat \cdot (h_t/2 \cdot (f(t_1)+f(t_0)))$
\FOR{$l = 1,2,\cdots k $} 
\STATE $q(t_{l+1}) = invmat \cdot (q(t_l) + h_t/2( A q(t_{l}) + f(t_l)v+f(t_{l+1})v))$
\ENDFOR
\end{algorithmic} 
\end{algorithm}

Note that we have used MATLAB's invertion algorithm, and not the "$\backslash$" operator. This is because performing one matrix inversion and $k$ matrix vector products was faster than using the "$\backslash$" operator $k$ times.
%!!!!!!!!!!!!!!forklare hvorfor matrisen er invertert?!!!!!!!!!!!!!!!!!!!!!!!!\\

%%%%%%%%%%%%%%%%%%%%%%%%%%%%%%%%%%%%%%%%%%%%%%%%%%%%%%%%%%%%%%%%%%%%%%%%%%%%%%%%%%%%%%%%%%%%%%%%%%%%%%%%%%%%%%%%%%%%%%
\section{Measurements and computers} \label{sec:not}
%%%%%%%%%%%%%%%%%%%%%%%%%%%%%%%%%%%%%%%%%%%%%%%%%%%%%%%%%%%%%%%%%%%%%%%%%%%%%%%%%%%%%%%%%%%%%%%%%%%%%%%%%%%%%%%%%%%%%%
%!!!!!!!!!!!!!!!!!!!Skrive mer om hva vi er på jakt etter av resultater i forhold til $p$!!!!!!!!!!!!!!!!!!!\\
%!!!!!!!!!!!!!!!!!!!!!!!!!!!!!Skrive at vi bruker alle iterasjonene når vi bruker KPM!!!!!!!!!!!!!!!!!!!!!!!!!!!!!!!!\\
Algorithms \ref{alg:arnoldi}, \ref{alg:fullkry}, \ref{alg:restkry}, together with equation \ref{eqn:DI} was implemented as solvers for equation \ref{eqn:heat}, with varieties of algorithm \ref{alg:integrate} solving the differential equations. 

Each solver was implemented in two versions, one where $p$ was assumed separable, and one where $p$ was assumed to be non-separable. Parallel computations was only implemented for KPM with $p$ non-separable, because in this case we needed to solve $m$ independent problems, and not just one.\\ %as is the case when $p$ is separable. \\

All solvers and problems were implemented in MATLAB R2014b. The computer used runs Ubuntu 14.04 LTS with intel  i7-4770 CPU, and has 16 GB ram. \\

The parallel computations was implemented with MATLAB's commands \texttt{parpool} and \texttt{parfor}, see \cite{parpool} and \cite{parfor} for more information, \texttt{nP} denotes the number of processing units used. Speedup and parallel efficiency will be used to investigate the gain by using several processing units. Speedup is defined as
\begin{align*}
S_\texttt{nP} = \frac{\tau_1}{\tau_\texttt{nP}}
\end{align*}
and parallel efficiency as
\begin{align*}
\eta_\texttt{nP} = \frac{S_\texttt{nP}}{\texttt{nP}}
\end{align*}
where $\tau_\texttt{nP}$ is run time with \texttt{nP} processors, $S_\texttt{nP}$ is speedup with \texttt{nP} processors, and $\eta_\texttt{nP}$ is parallel efficiency for \texttt{nP} processors. Speedup measures how much faster a program runs with \texttt{nP} processors, ideal speedup is when $S_\texttt{nP} = \texttt{nP}$. Parallel efficiency measures how well each processor is used. Perfect parallel efficiency occurs when $\eta_\texttt{nP} = 1$.\\

Remember that \texttt{KPM}$(n)$ is the method where the $n$ first iterations of Arnoldi's algorithm is used to obtain $V_n$ and $H_n$, and that \texttt{KPM} is short for \texttt{KPM}$(m)$. Let $n$ be called a restart variable. The number of restarts needed for convergence in algorithm \ref{alg:restkry} is denoted by $\gamma$. The convergence criterion used in algorithm \ref{alg:restkry} is to stop restarting when the maximum absolute difference in $q_n(t)$ is less than a given tolerance $\delta$. The error is denoted as $\epsilon$ and is defined as the largest absolute difference between the correct solution and the approximation. \\

If nothing else is stated, assume that $\rho =40$, $k = 40$, $n = 1$, $\delta = 10^{-15}$, these numbers was chosen as large as possible, while still giving an answer in a timely manner. All timed results are averaged over 2 runs, preferably this should have been higher, but would be too time consuming. $A$ was implemented as a sparse matrix.\\
 
For separable $p$, computation time and error and how this scale with $\delta$ will be of interest. When $p$ is not separable; convergence, computation time and parallel gain is of interest. \\



%%%%%%%%%%%%%%%%%%%%%%%%%%%%%%%%%%%%%%%%%%%%%%%%%%%%%%%%%%%%%%%%%%%%%%%%%%%%%%%%%%%%%%%%%%%%%%%%%%%%%%%%%%%%%%%%%%%%%%
\section{Test problems} \label{sec:test}
%%%%%%%%%%%%%%%%%%%%%%%%%%%%%%%%%%%%%%%%%%%%%%%%%%%%%%%%%%%%%%%%%%%%%%%%%%%%%%%%%%%%%%%%%%%%%%%%%%%%%%%%%%%%%%%%%%%%%%
Two test problems are implemented for the separable case. Equation \eqref{eqn:sep1} is a symmetric problem, and separable for each variable. It is denote as \texttt{P1}. 
\begin{equation} \label{eqn:sep1}
\begin{aligned}
 u(t,x,y)&= \frac{t}{t+1} x(x-1)y(y-1) \\
 f_1(t)&=\frac{1}{(t+1)^2} & g_1(x,y)&= x(x-1)y(y-1) \\
 f_2(t) &= \frac{-t}{t+1} & g_2(x,y)& = 2x(x-1) +2y(y-1)
 \end{aligned}
\end{equation}
Equation \eqref{eqn:sep2} is not symmetric, and non-separable for $x$ and $y$, it also has a combination of polynomials and exponential functions, just to make it test a more general case. 
This problem will be denoted as \texttt{P2}.\\
\begin{equation} \label{eqn:sep2}
\begin{aligned}
 u(t,x,y)=& e^{xy}y(y-1) \sin( \pi x)t \cos(t)& \\
 f_1(t) =& \cos(t)-t \sin(t)  & g_1(x,y) =&e^{xy}y(y-1) \sin( \pi x)\\
 f_2(t) =& -t \cos(t) \\ g_2(x,y) =&(y-1)y^3e^{xy} \sin ( \pi x)
 \end{aligned}
\end{equation}
\begin{equation*}
\begin{aligned}
&+e^{xy}(x^2(y-1)y+x(4y-2)+2) \sin( \pi x) \\&+2 \pi (y-1) y^2 e^{xy} \cos( \pi x)- \pi^2 (y-1)y e^{xy} \sin( \pi x )
 \end{aligned}
\end{equation*}

To obtain the solutions, solve for $f_i(t) g_i(x,y)$, $i = 1,2 $ and add the solutions together. Clearly parallel computations could be used to solve these, but this will not be done in this text. The reason for this is that MATLAB uses a significant amount of time to prepare parallel computation, often much more time than the computation itself.\\

\texttt{P1} will also be used in the non-separable case with $p(t) = f_1(t) g_1(x,y) + f_2(t) g_2(x,y)$.
One additional test problem is implemented for non-separable $p$, this is a symmetric problem consisting of both polynomial and exponential functions, it is given in equation \eqref{eqn:non1}, and will be denoted as \texttt{P3}.
\begin{equation} \label{eqn:non1}
\begin{aligned}
 u(t,x,y) = & \sin(x y t) (x-1) (y-1)\\
 p(t,x,y) = & t^2 (x-1) (y-1) y^2 \sin(t x y)\\ & -2 t (y-1) y \cos(t x y)+(x-1) x (y-1) y \cos(t x y)\\ & -t (x-1) x (2 \cos(t x y)-t x (y-1) \sin(t x y))
\end{aligned}
\end{equation}
%There is no other reason the problems for non-separable $p$ is not symmetric, except for laziness.
Even though both the test problems for non-separable $p$ are symmetric, the method works equally well on non-symmetric problems. 
%!!!!!!!!!!!!!!!Skrive at ting fungerer for ikke symetriske problemer også, og at det var tilfedligheter som gjorde at det ble slik det ble!!!!!!!!!!!!!!!!!!!!!!!!!!!\\