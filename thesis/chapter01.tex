%%%%%%%%%%%%%%%%%%%%%%%%%%%%%%%%%%%%%%%%%%%%%%%%%%%%%%%%%%%%%%%%%%%%%%%%%%%%%%%%%%%%%%%%%%%%%%%%%%%%%%%%%%%%%%%%%%%%%%
\chapter{Introduction}%%%%%%%%%%%%%%%%%%%%%%%%%%%%%%%%%%%%%%%%%%%%%%%%%%%%%%%%%%%%%%%%%%%%%%%%%%%%%%%%%%%%%%%%%%%%%%
%%%%%%%%%%%%%%%%%%%%%%%%%%%%%%%%%%%%%%%%%%%%%%%%%%%%%%%%%%%%%%%%%%%%%%%%%%%%%%%%%%%%%%%%%%%%%%%%%%%%%%%%%%%%%%%%%%%%%%

The aim of this report is to investigate how the Krylov projection method (KPM)  can be used to solve linear differential equations on the form $q'(t)=Aq(t)+f(t)$. These problems arise for example when discretizing time dependent, linear partial differential equations with the method of lines, and have therefore a wide range of applications. 
KPM is an orthogonal projection technique, the method is explained in detail in section \ref{sec:krylov}. See also \textit{A krylov projection method for systems of ODEs} by E. Celledoni and I. Moret\cite{elena} for more information about the method. \\

%We use Arnoldi's method which allows us to construct an orthonormal basis for a given Krylov subspace, this gives rise to two slightly different methods, presented in detail in section \ref{sec:fullKPM} and \ref{sec:rest} respectively.

In this report the heat equation will be solved with KPM. Convergence and computation time for KPM will be compared with an alternative solution technique. The alternative will be presented in section \ref{sec:DM}.
The heat equation will be stated here with boundary conditions for future references: \\
%The heat equation is 
\begin{equation} \label{eqn:heat}
\begin{aligned}
\frac{\partial u(t,x,y)}{\partial t} - \nabla^2 u(t,x,y) &= p(t,x,y) & t \in [0,T]\\
u(0,x,y) &= 0 \\
u & = 0 			&\text{ on } \partial [0,L] \times [0,L]\cite{heat},
\end{aligned}
\end{equation}
where $p(t,x,y)$ is a smooth function, and $u(t,x,y)$ is the solution sought.\\

Assume that the right hand side of equation \eqref{eqn:heat} is separable, so that $p(t,x,y) = f(t)g(x,y) $, how this problem can be solved if this is not the case will be explained in section \ref{sec:nonsep}.
Given the vector $v = [v_1,v_2, \cdots, v_m] \in \mathbb{R}^m $ with elements $ g(x_i,y_j)$, $x_i,y_j \in [0,L]$ and the matrix $A$ as an approximation of the Laplacian, the space-discrete version of the heat equation can be written as

\begin{equation} \label{eqn:numheat}
\begin{aligned}
q'(t) -Aq(t) &= f(t) v, \qquad & t \in [0,T] \\
q(0) &= 0               
\end{aligned}
\end{equation}
where $A$ is an $m \times m$ matrix assumed to be time independent, $f(t)$ is continuous on $[0,T]$, $v \in \mathbb{R}^m$, $q(t)$ is the sought solution vector on $[0,T]$. Note also that $f(t)$ is a scalar function, so that $f(t) v = [f(t)v_1,f(t)v_2, \cdots,f(t) v_m] $.
The solution is 
\begin{equation*}
q(t) = \int \limits_0^t \exp(A(t-s))f(s)v ds.
\end{equation*}
More details about $A$, $x_i$, $y_i$ and $q(t)$ will be given in section \ref{sec:imp}.
