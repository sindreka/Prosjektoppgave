%%%%%%%%%%%%%%%%%%%%%%%%%%%%%%%%%%%%%%%%%%%%%%%%%%%%%%%%%%%%%%%%%%%%%%%%%%%%%%%%%%%%%%%%%%%%%%%%%%%%%%%%%%%%%%%%%%%%%%
\chapter{Krylov subspace and methods}%%%%%%%%%%%%%%%%%%%%%%%%%%%%%%%%%%%%%%%%%%%%%%%%%%%%%%%%%%%%%%%%%%%%%%%%%%%%%%%%
%%%%%%%%%%%%%%%%%%%%%%%%%%%%%%%%%%%%%%%%%%%%%%%%%%%%%%%%%%%%%%%%%%%%%%%%%%%%%%%%%%%%%%%%%%%%%%%%%%%%%%%%%%%%%%%%%%%%%%
%This section will be concerned with deriving KPM.
\label{sec:krylov}
This section will show the mathematics needed to obtain KPM.
We present the Krylov subspace in section \ref{sec:subspace}, derive KPM for the heat equation in section \ref{sec:fullKPM} and \ref{sec:rest}. In section \ref{sec:nonsep} we show how we can use KPM when $p$ is not separable. 
 %in section \ref{sec:nonsep}.


%%%%%%%%%%%%%%%%%%%%%%%%%%%%%%%%%%%%%%%%%%%%%%%%%%%%%%%%%%%%%%%%%%%%%%%%%%%%%%%%%%%%%%%%%%%%%%%%%%%%%%%%%%%%%%%%%%%%%%
\section{Krylov subspace} \label{sec:subspace}
%%%%%%%%%%%%%%%%%%%%%%%%%%%%%%%%%%%%%%%%%%%%%%%%%%%%%%%%%%%%%%%%%%%%%%%%%%%%%%%%%%%%%%%%%%%%%%%%%%%%%%%%%%%%%%%%%%%%%%
The Krylov subspace is the space $W_n (A,v) = \{v,Av, \cdots, A^{n-1}v\} = \{v^1,v^2,\cdots v^n\} $, where $n \leq m$. %Define $\nu$ as the $W_\nu(A,v)$ span $A$. 
The vectors $v_i$ together with $h_{i,j} = v_i^\top Av_j$, are found by using Arnoldi's algorithm, shown in algorithm \ref{alg:arnoldi}. Letting $V_n$ be the $m \times n$ matrix consisting of column vectors $\{v^1,v^2,\cdots,v^n \}$ and $H_n$ be the $n \times n$ upper Hessenberg matrix containing all elements $(h_{i,j})_{i,j=1,\cdots,n}$, the following holds \cite{saad}
\begin{align}\
AV_n & = V_n H_n + h_{n+1,n}v_{n+1}e^\top_n \label{eqn:prop1} \\
V^{\top}_n AV_n &= H_n \label{eqn:prop2} \\
v_i^{\top} v_j &= \delta_{i,j} \label{eqn:prop3}
\end{align}
Here, $e_n$ is the $n$th canonical vector in $\mathbb{R}^n$ and $\delta_{i,j}$ is Kronecker's delta.\\




\begin{algorithm} 
\begin{algorithmic} \caption{Arnoldi's algorithm\cite{saad}} \label{alg:arnoldi}  
\STATE Start with $A$, $v^1 = v/\|v \|_2$, n
\FOR{$j = 1,2,\cdots n $} 
   \STATE Compute $h_{i,j} = \langle Av^j,v^i \rangle $ for $i = 1,2,\cdots j$
    \STATE Compute $w_j := A v^j - \Sigma_{i=1}^{j} h_{i,j}v^i $
    \STATE $h_{j+1,j} = \| w_j \|_2$
    \IF{$h_{j+1,j} = 0$} 
        \STATE\textbf{STOP}
    \ENDIF 
   \STATE $v^{j+1} = w_j/h_{j+1,j}$
\ENDFOR
\end{algorithmic} 
\end{algorithm}
%Let us assume that $\nu \leq m$ is the index such that $W_v(A,v)=\mathbb{R}^m$.
%Let us define $\nu  \leq m$ as the constant where $W_\nu(A,v)$ spans $A$, meaning we run Arnoldi's algorithm until the stopping criterium is met.
%%%%%%%%%%%%%%%%%%%%%%%%%%%%%%%%%%%%%%%%%%%%%%%%%%%%%%%%%%%%%%%%%%%%%%%%%%%%%%%%%%%%%%%%%%%%%%%%%%%%%%%%%%%%%%%%%%%%%%
\section{Krylov projection method} \label{sec:fullKPM}
%%%%%%%%%%%%%%%%%%%%%%%%%%%%%%%%%%%%%%%%%%%%%%%%%%%%%%%%%%%%%%%%%%%%%%%%%%%%%%%%%%%%%%%%%%%%%%%%%%%%%%%%%%%%%%%%%%%%%%

Let $z(t) = [z_1(t), z_2(t), \cdots, z_m(t)] \in \mathbb{R}^m $ be the vector satisfying $q(t) = V_m z(t)$, where $q(t)$ is from equation \eqref{eqn:numheat}, and $V_m$ is obtained by running algorithm \ref{alg:arnoldi} with $n = m$. 
We derive KPM by writing this into \eqref{eqn:numheat}, that is
\begin{equation*}  \begin{aligned} \label{eqn:KPMtemp1}
V_m z'(t) - A V_m z(t) &= f(t) v \\
z(0)& = 0
\end{aligned} \end{equation*}
Multiplying by $V_m^{\top}$ and using equation \eqref{eqn:prop2} gives
\begin{equation*} 
\begin{aligned} \label{eqn:KPMtemp2}
z'(t)-H_m z(t) &= f(t) V_m^{\top}  v  \\
z(0)& = 0
\end{aligned}
\end{equation*}
Using equation \eqref{eqn:prop3} and $v = \|v \|_2 v^1 $, we get
\begin{equation} 
\begin{aligned} \label{eqn:krylov}
z'(t) -H_m z(t) &=  \|v \|_2 e_1 f(t)\\
z(0)& = 0
\end{aligned}
\end{equation}
By solving equation \eqref{eqn:krylov} for $z(t)$ and calculating $ q(t) = V_m z(t) $ we obtain the solution. A step by step description is given in algorithm \ref{alg:fullkry}. 
%\end{proof}
 We will denote the method KPM.\\
\begin{algorithm}
\begin{algorithmic} \caption{The Krylov projection method} \label{alg:fullkry} 
\STATE Start with $A$,$f(t)$ and $v$.
\STATE Compute $[V_m ,H_m] = \texttt{arnoldi}(A,v)$
\STATE Solve $  z'(t) = H_m z + f(t) \| v \|_2 e_1  $ for $z$
\STATE $ q_m (t) \leftarrow  V_m z(t) $
\end{algorithmic} 
\end{algorithm}

Let us now consider the residual of equation \eqref{eqn:numheat} at $q_n(t) = V_n z (t)$, that is
\begin{equation*}
r_n(t) = f(t) v - q_n'(t) +Aq_n(t)
\end{equation*}
Since
\begin{equation*}
r_n(t) = f(t)v -V_n z'(t) + A V_n z(t)
\end{equation*}
using equation \eqref{eqn:prop1} and \eqref{eqn:krylov} we get
\begin{equation} \label{eqn:rn}
r_n(t) = h_{n+1,n}e_n^\top z(t) v_{n+1}
\end{equation}

Since $h_{n+1,n} = 0$ for some $n \leq m$, this shows the finite termination of the procedure.


%%%%%%%%%%%%%%%%%%%%%%%%%%%%%%%%%%%%%%%%%%%%%%%%%%%%%%%%%%%%%%%%%%%%%%%%%%%%%%%%%%%%%%%%%%%%%%%%%%%%%%%%%%%%%%%%%%%%%%
\section{Restarting the Krylov projection method} \label{sec:rest}
%%%%%%%%%%%%%%%%%%%%%%%%%%%%%%%%%%%%%%%%%%%%%%%%%%%%%%%%%%%%%%%%%%%%%%%%%%%%%%%%%%%%%%%%%%%%%%%%%%%%%%%%%%%%%%%%%%%%%%
If $n < m$ so that $h_{n+1,n} \neq 0$, we need to restart the procedure described above. Consider first the following equation
\begin{equation}\label{eqn:restkry}
\begin{aligned}
 (q-q_n)'(t) -A (q-q_n)(t) &= r_n \\
(q-q_n)(0)& = 0
\end{aligned}
\end{equation}
where $r_n$ is as in equation \eqref{eqn:rn}. If we can solve this equation for $(q-q_n)$, we can improve the approximation of $q$ via iterative refinement.

Equation \eqref{eqn:restkry} is of the same form as equation \eqref{eqn:krylov}. We derive KPM as we did before, by writing $q(t) = V_m z(t)$ and $q_n = \tilde{V}_n \tilde{\zeta} $. Let $\tilde{V}_n$ be the $m \times m$ matrix with the first $n$ columns equal to the first $n$ columns of $V_m$, and all additional elements equal to zero. The first $n$ rows of  $\tilde{\zeta}$ is equal to $\zeta$, where $q_n = V_n \zeta$. The rest of the elements in $\tilde{\zeta}$ are equal to zero.
We then get
\begin{equation*}
\begin{aligned}
 (V_m z-\tilde{V}_n \tilde{\zeta})'(t)-A (V_m z-\tilde{V}_n \tilde{\zeta})(t) &=  h_{n+1,n}e_n^\top \tilde{\zeta}(t) v_{n+1}  \\
(z-\tilde{\zeta})(0)& = 0 
\end{aligned}
\end{equation*}
Multiplying by $V_m^{\top}$ and using equation \eqref{eqn:prop2} gives
\begin{equation*}
\begin{aligned}
 (z-\tilde{\zeta})'(t)-\tilde{H_n} (z-\tilde{\zeta})(t) &= V_m^\top h_{n+1,n}e_n^\top \tilde{\zeta}(t) v_{n+1}  \\
(z-\tilde{\zeta})(0)& = 0
\end{aligned}
\end{equation*}
Let $\tilde{\xi}(t) = (z-\tilde{\zeta})(t)$, and simplify
\begin{equation*} 
\begin{aligned}
 \tilde{\xi} '(t) -\tilde{H_n} \tilde{\xi}(t) &= h_{n+1,n}e_n^\top \tilde{\zeta} (t)  \\
\tilde{\xi}(0)& = 0
\end{aligned}
\end{equation*}
If we drop all except the $n$ first rows of $\tilde{\xi}(t)$ we are left with
\begin{equation}\label{eqn:restkry2}
\begin{aligned}
 \xi '(t) -H_n \xi(t) &= h_{n+1,n}e_n^\top \zeta (t)  \\
\xi(0)& = 0
\end{aligned}
\end{equation}

Each restart we generate a new Krylov subspace $W_n(A,v^{n+1})$, solve equation \eqref{eqn:restkry2} for $\xi(t)$ and approximate the solution $q$ by $ q_n =  V_n\xi(t)$. By summing together $q_n$, we converge towards $q$. Note that the current value of $\zeta(t)$ equals the previous value of $\xi(t)$, and that $h_{n+1,n}$ is from the previous $H_n$. See algorithm \ref{alg:restkry} for a step by step description. We will call $n$ a restart variable, and denote the method with KPM$(n)$.
\begin{algorithm}
\begin{algorithmic} \caption{The Krylov projection method with restart} \label{alg:restkry} 
\STATE Start with $A$,$f(t)$,$v$, $n$ and $i = 0$
\STATE Compute $[V_n,H_n,h_{n+1,n}^i,v^{n+1}] = \texttt{arnoldi}(A,v)$
\STATE Solve $  z' = H_n z + f(t) \| v \|_2 e_1  $ for $z$
\STATE $ q_n \leftarrow  V z $
\STATE $\xi_i \leftarrow z$
\WHILE{convergence criterion not satisfied} 
    \STATE $i \leftarrow i + 1$
    \STATE Compute $[V_n,H_n,h_{n+1,n}^i,v^{n+1}] = \texttt{arnoldi}(A,v^{n+1},n)$
    \STATE Solve $ \xi_i'(t) = H_n \xi_i(t) + h_{n+1,n}^{i-1}e_n^\top \xi_{i-1}(t)  $ for $\xi_i$
    \STATE $ q_n(t) \leftarrow q_n + V_n \xi_i(t) $
\ENDWHILE
\end{algorithmic} 
\end{algorithm}

%%%%%%%%%%%%%%%%%%%%%%%%%%%%%%%%%%%%%%%%%%%%%%%%%%%%%%%%%%%%%%%%%%%%%%%%%%%%%%%%%%%%%%%%%%%%%%%%%%%%%%%%%%%%%%%%%%%%%%
\section{When $p$ is not seperable} \label{sec:nonsep}
%%%%%%%%%%%%%%%%%%%%%%%%%%%%%%%%%%%%%%%%%%%%%%%%%%%%%%%%%%%%%%%%%%%%%%%%%%%%%%%%%%%%%%%%%%%%%%%%%%%%%%%%%%%%%%%%%%%%%%
If we let $P(t)$ be a vector consisting of elements $p(t, x_i, y_j)$, so that $P(t) = [P_1(t),P_2(t),\cdots, P_m(t)]$, and write equation \eqref{eqn:numheat} as

\begin{equation}
\begin{aligned}
q'_j(t) -A q_j(t) &= P_j(t) e_j \\
q_j(0) &= 0\\
q(t) &= \sum \limits_{i = 1}^m q_j(t)
\end{aligned}
\end{equation}
where $e_j$ is the $j$th canonical vector in $\mathbb{R}^{m}$, we can solve the original equation without requiring separability. 
An important thing to note here is the need for parallel processing power since we need to solve $m$ problems and not just one.
